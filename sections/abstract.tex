\chapter*{\Large \center Resumé}
Dans le cadre de ce projet de fin d’études, nous avons conçu, développé et déployé une plateforme web baptisée \textbf{Swift Helpers}, destinée à mettre en relation des clients avec des prestataires de services (helpers) dans un environnement numérique fluide, intuitif et sécurisé.

Le projet a été mené selon une démarche agile, découpée en plusieurs sprints successifs. Chaque sprint a couvert une fonctionnalité majeure du système : gestion des comptes clients, gestion des prestataires, traitement des ordres de mission, supervision administrative, et enfin déploiement de l’ensemble du système dans un environnement conteneurisé à l’aide de Docker.

Sur le plan technique, la solution repose sur un backend développé avec \textbf{Django REST Framework}, un frontend en \textbf{Angular}, une base de données \textbf{PostgreSQL}, et un déploiement orchestré par \textbf{Docker} et \textbf{NGINX}. Ce socle technologique garantit modularité, performance et maintenabilité.

Ce rapport retrace toutes les étapes de réalisation du projet, de l’analyse initiale des besoins jusqu’à la mise en production, tout en mettant en évidence les choix techniques, les modèles de données, les diagrammes UML et les interfaces utilisateur développées.

\noindent Mots-clés: Développement web, Swift Helpers, plateforme de services, Django, Angular, PostgreSQL, API REST, Docker, architecture logicielle, méthode agile, déploiement.\\[1mm]
\rule[1em]{38em}{0.5pt}