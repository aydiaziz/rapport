\chapter*{\Large \center Introduction générale}

Dans un monde en constante évolution numérique, les plateformes de services à la demande se multiplient pour répondre aux besoins croissants d’efficacité, de flexibilité et de réactivité. Qu’il s’agisse de mise en relation entre particuliers ou de solutions destinées à des structures professionnelles, ces systèmes doivent être pensés pour offrir à la fois une expérience utilisateur fluide, une architecture technique robuste et une évolutivité assurée.

C’est dans ce contexte qu’intervient le projet \textbf{Swift Helpers}, développé dans le cadre de mon projet de fin d’études au sein de la spécialité Génie Logiciel de l’école d’ingénieurs ESPRIT. Ce projet vise à concevoir, développer et déployer une application web complète permettant la gestion d’un réseau de prestataires de services, avec une attention particulière portée à la simplicité d’utilisation, à la modularité technique et à l’automatisation des processus métier.

Le développement de la plateforme s’inscrit dans une démarche agile, structurée en sprints, couvrant toutes les étapes de réalisation : de l’analyse des besoins à l’implémentation des fonctionnalités, jusqu’au déploiement sur un environnement de production Dockerisé. Le système repose sur un backend \textbf{Django REST}, un frontend \textbf{Angular}, une base de données \textbf{PostgreSQL}, et une infrastructure conteneurisée à l’aide de \textbf{Docker}.

Ce projet m’a offert l’opportunité d’appliquer concrètement les compétences acquises tout au long de ma formation, que ce soit en architecture logicielle, en développement fullstack, en conception orientée objet ou en gestion de projet. Il représente également une première immersion dans les exigences réelles d’un système exploitable par des utilisateurs finaux.

Ce rapport présente, dans un premier temps, le contexte général du projet et son périmètre fonctionnel. Il détaille ensuite les différentes phases de développement, organisées par sprint, avant de conclure par le processus de déploiement, les choix techniques effectués, et les perspectives d’évolution du système.
